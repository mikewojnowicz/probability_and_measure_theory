\documentclass{article} % For LaTeX2e

%STANDARD PREAMBLE
%https://tex.stackexchange.com/questions/68821/is-it-possible-to-create-a-latex-preamble-header
\usepackage{/Users/mwojno01/Research/Learning/latex_preamble/preamble}

%%% 
% SPECIFIC TO THIS DOCUMENT
%%%

% SIGMA-FIELDS 
% Sigma fields (text)
\renewcommand{\sf}{$\sigma$-field}
\newcommand{\sfs}{$\sigma$-fields}


% SET FUNCTIONS 
%Finitely additive set functions
\newcommand{\fasf}{\tilde{\mu}_0}
% Signed measure
\newcommand{\signedmu}{\wt{\mu}}


% ENUMERATE BUT WITH LETTERS
\newenvironment{alphabate}
    {\begin{enumerate}[label=\alph*)]}
	{\end{enumerate} }
    
\begin{document}


\title{Notes on Ash's Probability and Measure Theory} 
\maketitle
\tableofcontents
\newpage 

\section{Section 1.1: Some notes on set theory}

\begin{definition}
If $A_1 \subset A_2 \subset ...$ and $\cup_{n=1}^\infty A_n = A$, we say that the $A_n$ form a \textbf{increasing} sequence of sets with limit $A$ or that the $A_n$ increase to $A$; we write $A_n \uparrow A$.  If $A_1 \supset A_2 \supset ... $ and  	$\cap_{n=1}^\infty A_n = A$, we say that the $A_n$ form a \textbf{decreasing} sequence of sets with limit $A$ or that the $A_n$ decrease to $A$; we write $A_n \downarrow A$.
\end{definition}


Now some remarks on representing unions as disjoint unions. 

\begin{remark}
If $A_1,A_2,...$ are subsets of some set $\Omega$, then
\begin{align*} 
\bigcup_{n=1}^\infty A_n = \bigcupdot_{n=1}^\infty \bigg(A_n \cap A_{n-1}^c \cap ... \cap A_1^c \bigg) 	
\labelit \label{eqn:union_as_disjoint_union}
\end{align*}

In other words, any union can be re-represented as a disjoint union. This is useful because measures are countably additive on disjoint sets, so we prefer to work with collections of disjoint sets.
\label{rk:rerepresenting_unions_as_disjoint_unions}
\end{remark}

\begin{remark}
If $A_n \uparrow A$, then \eqref{eqn:union_as_disjoint_union} becomes
\begin{align*}
	\bigcup_{n=1}^\infty A_n = \bigcupdot_{n=1}^\infty \bigg( A_n - A_{n-1} \bigg) 
\labelit \label{eqn:union_as_disjoint_union_for_increasing_sequences}
\end{align*}
This is because $A_{n-1} \subset A_{n}$, so $A_{n-1}^c \supset A_{n}^c$ by contraposition.	
\end{remark}

\section{Section 1.2: Fields, \sfs, measures}

\subsection{Sec 1.2.1-1.2.2: Fields and \sfs}

Fields and \sfs\ are important because they are the domain of measures.  Here are some definitions.

\begin{definition}
Let $\F$ be a collection of subsets of a set $\Omega$.  Then $\F$ is called a \textbf{field} (or \textit{algebra})  if 

\begin{enumerate}[label=\alph*)]
	\item $\Omega \in \F$ 
	\item If $A \in \F$, then $A^c \in \F$.
	\item If $A_1,...A_n \in \F$ then $\cup_{i=1}^n A_i \in \F$. 
\end{enumerate}
that is, if $\Omega \in \F$ and $\F$ is closed under complementation and finite unions.
\label{def:field}	
\end{definition}

\begin{definition}
Let $\F$ be a collection of subsets of a set $\Omega$.  Then $\F$ is called a \textbf{sigma-field} (or \textit{sigma-algebra}) if it satisfies Definition \ref{def:field} after replacing condition c) with

\begin{enumerate}
	\item[c')] If $A_1,A_2, ... \in \F$ then $\cup_{i=1}^\infty A_i \in \F$. 
\end{enumerate}
that is, if $\Omega \in \F$ and $\F$ is closed under complementation and \textit{countable} unions.
\label{def:sigma-field}	
\end{definition}

\begin{example}
$\F =\set{\emptyset, \Omega}$ is the smallest \sf\ on $\Omega$. 
\end{example}

\begin{example}
	$\F =2^\Omega$, i.e. the set of all subsets of $\Omega$, is the largest \sf\ on $\Omega$.
\end{example}

\begin{example}
If $A \in \Omega$ is non-empty, then $\F = \set{\emptyset, A, A^c, \Omega}$ is the smallest \sf\ containing $A$.
\end{example}

\begin{notation}
If $\C$ is a class of sets, the smallest \sf\ containing the sets of $\C$ is written as $\sigma(\C$).  This is sometimes called the \textit{minimal \sf\ over $C$} or the \textit{\sf\ generated by $C$}. 
\end{notation}
	
\begin{question}
Let $A_1,...,A_n$ be subsets of $\Omega$.  Describe $\sigma(\set{A_1,...,A_n})$, the smallest \sf\ containing $A_1,...,A_n$.  Also describe the number of sets in $\F$.   \textit{This is Ash's Problem 1.2.8.  For answer, see GoodNotes.}	
\end{question}

\begin{example} What is an example of a collection that is a \textit{field}, but not a $\sigma$-\textit{field}?  

Let $\Omega=\R$ and $\F_0 = \set{\text{finite disjoint unions of right semi-closed intervals } (a,b], a \neq b}$.  Then $\F_0$ is a field, as can be easily verified.\footnote{By convention, we also count $(a, \infty)$ as right semi-closed for $-\infty\leq a < \infty$, which is necessary for the \sf\ to be closed under complements.}   But $\F_0$ is \underline{not} a \sf.  Note that if $A_n = (-\frac{1}{n},0]$, then $\bigcap_{n=1}^\infty A_n = \set{0}  \not\in \F_0$.
\label{ex:field_of_finite_disjoint_unions_of_lsc_intervals}
\end{example}

\begin{remark}
A \sf\ can also be described as a field that is closed under limits of increasing sequences.  For if $A_n \in \F$ and $A_n \uparrow A$, then A is a countable union of sets in $\F$ by definition.  Conversely, if $A = \cup_{n=1}^\infty A_n$, then set $B_N := \cup_{n=1}^N A_n$ and $B_N \uparrow A$.	
\end{remark}


\subsubsection{``Good sets" strategy}

Ash says that there is a type of reasoning that occurs so often in problems involving \sfs that it deserves explicit mention.  It is called the \textit{good sets strategy}.   Suppose you want to show that all members of a $\sigma$-algebra   $\F$ have some property $P$.  Define ``good sets" as those that satisfy the property
\[ \G := \{ G \in \F : G \text{ has property } P \} \]
The strategy is then to simply
\begin{enumerate}
\item Show $\G$ is a $\sigma$-algebra 
\item Show $\G$ contains some class $\C$ such that $\F = \sigma(\C)$	
\end{enumerate}


Then you're done!  

Why does this work?

\begin{align*}
& \quad \C \subset \G &&	\text{by 2}\\
&\implies \sigma(\C) \subset \sigma(\G) &&  \\
&\implies \F \subset \G && \text{by 1,2} \\
& \text{Yet $\G \subset \F$ by definition of $\G$.} && \\
& \text{So $\G = \F$.} && \\
& \text{ So all sets in $\F$ are good.} && \\
\end{align*}

In the text, Ash uses this strategy to show that if $\C$ is a class of subsets of $\Omega$, and $A \in \Omega$, then

\[ \explaintermbrace{take minimal sigma field first, then intersect}{\sigma_\Omega(\C) \cap A} = \explaintermbrace{intersect first, then take minimal sigma-field}{\sigma_A(\C \cap A)} \]

 For another application, see handwritten homework exercises.	

\subsection{Sec 1.2.3-1.2.4: Measures}




\begin{definition}
A \textbf{measure} on a \sf\ $\F$ is a non-negative, extended real-valued function $\mu$ on $\F$ such that whenever $A_1, A_2, ...$ form a finite or countably infinite collection of disjoint sets in $\F$, we have countable additivity; that is,
\[ \mu \bigg( \bigcupdot_n A_n \bigg) = \ds\sum_n \mu(A_n) \]
\label{def:measure}	
\end{definition}

\begin{definition}
A \textbf{probability measure} is a measure (Definition \ref{def:measure}) where $\mu(\Omega)=1$.
\label{def:prob_measure}		
\end{definition}

\begin{remark}
Ash additionally assumes that a measure does not take $\mu(A) = \infty$ or $\mu(A) = -\infty$ for all $A \in \F$.  From this, we automatically obtain $\mu(\emptyset)=0$. For $\mu(A) < \infty$ for some $A$, and by considering the sequence $A, \emptyset, \emptyset, ...$, we have that $\mu(\emptyset)=0$ by countable additivity.   	
\end{remark}

\begin{example}
Let $\Omega = \set{x_1,x_2,...}$ be a finite or countably infinite set.  Let $p_1, p_2,...$ be non-negative reals.  Let $\F = 2^\Omega$.  Define
\[\mu(A) = \ds\sum_{x_i \in A} p_i \quad \text{ for all } A \in \F\]
Then $\mu$ is a measure on $\F$. We might call it the ``point weighting" measure. 
\begin{itemize}
\item If $p_i \equiv 1 \; \forall \, i$, then $\mu$ is called the \textbf{counting measure}.
\item If $\sum_i p_i =1$, then $\mu$ is a probability measure.	
\end{itemize}
	
\end{example}


\begin{example}{\remarktitle{Lesbesgue measure}}
Define $\mu$ such that 
\[ \mu(a,b] = b-a \quad \forall \, a,b \in \R : b>a \]
As we will see in Section \ref{sec:extension_of_measures}, this requirement determines $\mu$ on a large collection of sets, the Borel Sets $\B(\R)$, defined as the smallest \sf\ of subsets of $\R$ containing all intervals $(a,b] \subset \R$.

We may alternately characterize $\B(\R)$ as the smallest \sf\ containing
\begin{itemize}
\item all intervals $[a,b], \; a,b \in \R$.
\item all intervals $(a,b), \; a,b \in \R$
\item all intervals $(a,\infty), \; a \in \R$.
\item  all intervals $[a,\infty), \; a \in \R$.
\item 	 all intervals $(-\infty,b), \; b \in \R$.
\item  all intervals $(-\infty,b], \; b \in \R$.
\item  all intervals $[a,b], \; a,b \in \R$.
\item all open sets of $\R$.\footnote{Recall that an open set is a countable union of open intervals.}
\item all closed sets of $\R$.\footnote{Recall that a set is open iff its complement is closed.}
\end{itemize}

To illustrate these equivalences, let us equate the first two conditions. That is, let us show that a \sf\ contains all open intervals $(a,b)$ iff it contains all right semi-closed intervals $(a,b]$.  To see this, simply note

\begin{align*}
(a,b] &= \bigcap_{n=1}^\infty \bigg(a, b+\frac{1}{n}\bigg) \\
	\intertext{and}
(a,b) &= \bigcup_{n=1}^\infty \bigg(a, b-\frac{1}{n}\bigg] 
\end{align*}

\end{example}

\begin{question}
The text gives another description of the Borel sets $\B(\R)$ as the smallest \sf\ containing $\F_0$, the field of disjoint unions of right semi-closed intervals $(a,b]$.  Can we make the same statement about the field of finite disjoint unions of left semi-closed intervals?
\end{question}

\subsection{Sec 1.2.5-1.2.6: Generalizations of measures, and their properties}

The text considers some generalizations of measures that can be obtained by restricting the domain to a field, by assuming merely finite additivity, or by allowing the range to be extended reals ($\bar{\R}$) instead of non-negative extended reals ($\bar{\R}_{\geq 0}$).  

\begin{remark}
The first two relaxations above often go together.
However, a  countably additive function can be defined on a \textit{field} (rather than \sf) if the condition is taken to hold whenever a countable union \textit{does} happen to still be in the field.  In my notes, I will simply things by assuming that countably additive functions are always defined on \sfs.
\label{rk:i_am_assuming_a_domain_based_on_the_type_of_additivity}
\end{remark}	



\begin{table}[!h]
\centering	
\begin{tabular}{rcc}
&\multicolumn{2}{c}{\textbf{Range}} \\
& \textbf{non-negative} & \textbf{signed} \\
\textbf{finitely additive}& $\mu_0$ & $\tilde{\mu}_0$ \\	
\textbf{countably additive}& $\mu$ measure  & $\tilde{\mu}$ signed measure \\

\end{tabular}
\caption{Notation for generalizations of measure (For assumed domain in each case, see Remark \ref{rk:i_am_assuming_a_domain_based_on_the_type_of_additivity}.)}
\label{tab:notation_for_generalizations_of_measure}
\end{table}

In Table \ref{tab:notation_for_generalizations_of_measure},  we introduce some notation to try to clarify more immediately when results hold. Note the relations\footnote{So, for example, if something holds for $\tilde{\mu}_0$, it holds for $\mu$.  A simple mnemonic is that adding stuff to the notation generalizes the function.}  
\[ \set{\mu} \subset \set{\mu_0}, \set{\tilde{\mu}} \subset \set{\tilde{\mu}_0}. \]
 
 Using the notation in Table \ref{tab:notation_for_generalizations_of_measure}, we rewrite Theorem 1.2.5 of the text:
 
 \begin{theorem}
 Let $\fasf$ be a finitely additive set function on the field $\F_0$.  Then
 \begin{enumerate}[label=\alph*)]
 \item \label{itm:first} $\fasf(\emptyset)=0$
 \item \label{itm:second} $\fasf(A \cup B) + \fasf (A \cap B) = \fasf(A) + \fasf(B)$ for all $A,B \in \F_0$.
 \item \label{itm:piece-and-difference} If $A,B \in \F_0$ and $B \subset A$, then   
  \[ \fasf(A) = \fasf(B) + \fasf(A-B)\quad \text{(piece-and-difference)} \] 
 So $\fasf(A) \geq \fasf(B)$ if $\fasf(A-B) \geq 0$. More generally, for non-negative set functions, we have
 \[ \mu_0 (A) \geq \mu_0 (B) \quad \text{(monotonicity)} \] 
 \item \label{itm:subadditivity} Subadditivity holds, i.e.
 \begin{align*}
 \mu_0 (\cup_{i=1}^n A_i)& \leq \sum_{i=1}^n \mu_0(A_i) \\
  \mu (\cup_{i=1}^\infty A_i)& \leq \sum_{i=1}^\infty \mu(A_i) \\
 \end{align*}
 \end{enumerate}
\label{thm:basic_properties_of_finitely_additive_set_functions}
 \end{theorem}
  
\begin{proof}
 We prove Theorem \ref{thm:basic_properties_of_finitely_additive_set_functions} (b).  The rest is an exercise for the reader (or see the text).
 
 First, we break things into disjoint pieces
 {\footnotesize 
\begin{align*}
A &= \bigg(A \cap B \bigg) \, \bigcupdot \, \bigg(A \cap B^c \bigg)	&& \implies \fasf(A) = \fasf (A \cap B) +  \fasf (A \cap B^c)  && (1) \\
B &= \bigg(A \cap B \bigg) \, \bigcupdot \, \bigg(A^c \cap B \bigg)	&& \implies \fasf(B) = \fasf (A \cap B) +  \fasf (A^c \cap B)  && (2) \\
A \cup B &= \bigg(A \cap B \bigg) \, \bigcupdot \, \bigg(A \cap B^c \bigg) \bigcupdot \, \bigg(A^c \cap B \bigg)	&& \implies \fasf(A \cup B) = \fasf (A \cap B) +  \fasf (A \cap B^c) + \fasf (A^c \cap B)   && (3) 
\end{align*}
}

Summing (1) and (2), we obtain
\[\fasf(A) + \fasf(B) = 2 \fasf(A \cap B) + \fasf(A \cap B^c) + \fasf(A^c \cap B). \]
We use (3) to simplify the RHS, and the result follows.
\end{proof}

\begin{remark}
In the proof of Theorem \ref{thm:basic_properties_of_finitely_additive_set_functions} (b), note that we use a common strategy -- breaking sets into disjoint pieces so that we can apply the assumed (finite or countable) additivity of the set function. 
\end{remark}

\begin{remark}
Being able to work with these generalizations will be important in Section \ref{sec:extension_of_measures} on extension of measures.  In particular, it will help us show that we can construct the Lesbesgue measure on the Borel sets.
\end{remark}

\subsection{Sec 1.2.7-1.2.8: Continuity of countably additive set functions}

Countably additive set functions have a basic continuity property. Continuity of measure is a special case. 

\begin{theorem}
Let $\signedmu$ be a countably additive set function on the \sf\ $\F$. Then

\begin{alphabate}
\item (continuity from below) If $A_1, A_2, ... \in \F$ and $A_n \uparrow A$, then $\signedmu(A_n) \to \signedmu(A)$ as $n \to \infty$. 
\item (continuity from above)  If $A_1, A_2, ... \in \F$, $A_n \downarrow A$, and $\signedmu(A_n)$ is finite, then $\signedmu(A_n) \to \signedmu(A)$ as $n \to \infty$. 
\end{alphabate}
\label{thm:continuity_of_countably_additive_set_functions}
\end{theorem}

\begin{proof}
We prove continuity from below, and leave continuity from above as an exercise to the reader (or see text).  First let us assume that all $\signedmu(A_n)$ are finite (*). Then 
\begin{align*}
A &= A_1 \cupdot A_2 - A_1 \cupdot A_3 - A_2 \cupdot ... && 	\text{ by } \eqref{eqn:union_as_disjoint_union_for_increasing_sequences} \\
\implies \signedmu(A) &= \signedmu(A_1) + \signedmu(A_2 - A_1) + \signedmu(A_3 - A_2) + ... && \text{(countable additivity)}\\ 
&= \signedmu(A_1) + \signedmu(A_2) - \signedmu(A_1) + \signedmu(A_3) - \signedmu(A_2) + ... && \text{(Theorem \ref{thm:basic_properties_of_finitely_additive_set_functions} \ref{itm:piece-and-difference}, (*) }\\
&= \ds\lim_{n \to \infty} \signedmu(A_n) 
\end{align*}
Now suppose $\signedmu(A_n) = \infty$ for some $n$.   So write 
\begin{align*}
A &= A_n \cupdot A- A_n && \text{(increasing sequence)}\\ 
\implies \signedmu(A) &= \signedmu(A_n) + \signedmu(A - A_n) && \text{(countable additivity)}\\  
&= \infty + \signedmu(A - A_n) 
\end{align*}

So $\signedmu(A)=\infty$.\footnote{Note that we cannot have $\signedmu(A - A_n)=-\infty$, because that would violate additivity.} Replace $A$ by $A_k$ for any $k \geq n$ to also find $\signedmu(A_k)=\infty$ for all $k \geq n$ and the result follows.

Finally suppose $\signedmu(A_n) = -\infty$ for some $n$. Then the result follows in the same way as for $\signedmu(A_n) = \infty$. 

\end{proof}

\begin{remark}
The logic of the proof of Theorem \ref{thm:continuity_of_countably_additive_set_functions} under the finiteness assumption is as follows.  First, we re-represent the union as a disjoint union (the form is particularly simple since the sets are increasing).  This allows us to apply countable additivity. Then we apply the piece-and-difference decomposition (and the subtraction is defined under the finiteness assumption). 	
\end{remark}

\begin{remark}
In proving Theorem \ref{thm:continuity_of_countably_additive_set_functions}  for the case where $\mu(A_n) = \infty$ for some $n$, it is tempting to make the simpler argument 
\begin{align*}
\mu(A) & \geq \mu(A_n) && \text{(monotonicity)}\\	
\mu(A_k) & \geq \mu(A_n) && \text{(monotonicity)}	 
\end{align*}
for $k \geq n$.  But recall from Theorem \ref{thm:basic_properties_of_finitely_additive_set_functions} that monotonicity only holds under non-negativity, and the theorem statement is more general, applying to \textit{signed} set functions as well. 
\end{remark}


%By , we have


 %we break up the limit of the increasing sequence into a disjoint union as follows	



 \section{Section 1.3: Extension of measures} \label{sec:extension_of_measures}
 



%\bibliography{references_ash_notes.bib}
%\bibliographystyle{unsrt}

\end{document}
